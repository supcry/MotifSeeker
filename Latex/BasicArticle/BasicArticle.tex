\documentclass[unicode,10pt,a4paper]{article}
\usepackage[UTF8]{inputenc}
\usepackage[T2A]{fontenc}
\usepackage[english,russian]{babel}
\usepackage{amsmath}
\usepackage{amsfonts}
\usepackage{amssymb}
\author{qwe}
\title{qweqwe}
\begin{document}
\textbf{раз} \textit{два} \underline{три} \begin{center}
четыре
\end{center} \begin{flushright}
пять
\end{flushright}
Проверка русского языка (его компилируемость латехом).

Далее, формулы:
$$
\alpha\beta\gamma\delta \neq \epsilon\varepsilon\zeta\eta
$$

Список ресурсов:
\begin{itemize}
\item Хромосомы - http://hgdownload.soe.ucsc.edu/goldenPath/hg19/chromosomes/
\item Разбивка по регионам -\\ http://genome.ucsc.edu/cgi-bin/hgFileUi?db=hg19\&g=wgEncodeUwDnase
\item Статья, которую Ваня дал для вхождения в суть -\\ http://www.sciencemag.org/content/suppl/2012/09/04/science.1222794.DC1/Maurano.SM.pdf
\end{itemize}
\end{document}